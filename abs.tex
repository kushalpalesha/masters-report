\providecommand{\keywords}[1]{\textbf{\textit{Keywords~~~}} #1}

Browser security has become a major concern especially due to web pages becoming
more complex. These web applications handle a lot of information including
sensitive information that may be vulnerable to attacks like data exfiltration,
cross-site scripting (XSS), etc. Most modern browsers have security mechanisms in
place to prevent such attacks but they still fall short in preventing more advanced
attacks like evolved variants of data exfiltration. Moreover, there is no standard
that is followed to implement security into the browser.
\par A lot of research has been done in the field of information flow security
that could prove to be helpful in solving the problem of securing the client side.
Policy-agnostic programming is a programming paradigm that aims to make implementation
of information flow security in real world systems more flexible. In this paper,
we explore the use of policy-agnostic programming on the client-side and how it
will help prevent common client-side attacks. We verify our results through a
client-side salary management application. We show a possible attack and how
our solution would prevent such an attack.\\
\keywords{Information flow security, policy-agnostic programming, faceted values}
