\chapter{Implementing policy-agnostic programming on the client side\label{chap:solution}}
We incorporated policy-agnostic programming into Javascript by adding constructs
defined for the Jeeves language as a subsystem\footnote{The code is available at:
\url{https://github.com/kushalpalesha/narcissus}}. We chose Narcissus~\cite{narc},
a Javascript interpreter written in Javascript, to create the proof of concept.
Narcissus was built by its developers to be able to prototype new language features
for Javascript.

\section{Implementing faceted values in Narcissus}
The implementation of faceted values is integral to implementing Jeeves. Our implementation
of faceted values derives heavily from the work done by Austin and Flanagan~\cite{Faceted}
and is inspired from the concept presented by Kerchove et al~\cite{Modular} for
``modular instrumentation'' of interpreters. They talk about how many dynamic analysis
approaches for information flow security have prototypes that are implemented in
very specific ways making it difficult to compare and reuse. They derive some
specific criteria to follow in order to achieve ``modular instrumentation'' using
the implementation~\cite{ZaphodFacets} of faceted values~\cite{Faceted} as a case
study. While we borrow a few ideas from this, our implementation does not follow
the criteria specified because achieving ``modular instrumentation'' would involve
non-trivial changes to the Narcissus interpreter making it out of scope for our problem.
What we did achieve is the untangling of the concerns of the core interpreter for
non-faceted evaluation from the concerns of faceted evaluation. This makes it easier
to relate principles of faceted evaluation with the implemented prototype and also
to extend or reuse it.

Our implementation of faceted values is independent from the core Narcissus interpreter.
This involved making the core interpreter modular to be able to modify existing
evaluation mechanisms to behave differently for faceted values. The Narcissus interpreter
has an execute function that consists of a switch case control flow that is set to
perform the appropriate set of operations based on the type of node identified by
the parser. Wherever there is need for faceted behavior, instead of making the core
interpreter code handle faceted behavior, we moved the part we would need to change
into a function and later override it to handle the faceted behavior.
Figure~\ref{fig:facetif} shows how this overriding is implemented for the [F-IF-SPLIT]
evaluation rule. Here, \texttt{BaseExecContext} is an object that stores a copy of all the
fucntions from the core interpreter which we need to override. \texttt{ExecutionContext}
is an object from the core interpreter that keeps track of the current flow of execution.
\texttt{FacetExecContext} is the \texttt{ExecutionContext} object extended with
the \textit{pc} to help keep track of the influence of public or private facets
on the current flow of execution. In the \texttt{evalIfBlock} function, notice we
call the base \texttt{evalIfBlock} function if \texttt{cond} is not a faceted value.
Otherwise, we call the \texttt{evaluateEach} function which implements the [FA-SPLIT]
rules from Figure~\ref{fig:appRules}. We override behavior of the rest of the functions
in \texttt{BaseExecContext} in a similar manner.
\lstset{
  language=javascript,
  frame=single,
  breaklines=true,
  basicstyle=\footnotesize\ttfamily,
  numbers=left,
  extendedchars=true,
  tabsize=2,
  showstringspaces=false
}
\begin{figure}
  \begin{lstlisting}
var BaseExecContext = {
  getValue : interpreter.ExecutionContext.prototype.getValue,
  putValue : interpreter.ExecutionContext.prototype.putValue,
  evalBinOp : interpreter.ExecutionContext.prototype.evalBinOp,
  evalUnaryOp : interpreter.ExecutionContext.prototype.evalUnaryOp,
  evalIfBlock : interpreter.ExecutionContext.prototype.evalIfBlock,
  evalDot : interpreter.ExecutionContext.prototype.evalDot,
  evalFunctionCall : interpreter.ExecutionContext.prototype.evalFunctionCall,
  runWhileLoop : interpreter.ExecutionContext.prototype.runWhileLoop
};

FacetExecContext.prototype.evalIfBlock = function(cond,thenPart,elsePart) {
  var execContext = FacetExecContext.current;
  if (cond instanceof FacetedValue) {
    evaluateEach(cond, function(v, x) {
      if (v) {
        interpreter.execute(thenPart, x);
      } else if (elsePart) {
        interpreter.execute(elsePart, x);
      }
    }, execContext);
  } else {
    BaseExecContext.evalIfBlock.call(this,cond,thenPart,elsePart);
  }
};
  \end{lstlisting}
  \caption{Independent implementation of faceted behavior for the ``if'' control flow}
  \label{fig:facetif}
\end{figure}

\begin{figure}
  \includegraphics[scale=0.5, frame]{images/appRules}
  \caption{Function application rules~\cite{FacetedJeeves}}
  \label{fig:appRules}
\end{figure}

\section{Implementing Jeeves \label{sec:jeeves}}
Once we had faceted values working, adding support for Jeeves constructs was pretty
straightforward. All of the Jeeves constructs are encapsulated in the prototype
of the \texttt{PolicyEnvironment} object which includes the [F-LABEL] and [F-RESTRICT]
evaluation rules as shown in Figure~\ref{fig:jeevesEval}. Every instance of the \texttt{PolicyEnvironment}
object has a \texttt{policyMap} that is used to store the `label':`policy' mapping.

\begin{figure}
  \centering
  \includegraphics[scale=0.3, frame]{images/jeevesEval}
  \caption{Evaluation semantics for Jeeves labels and policies~\cite{FacetedJeeves}}
  \label{fig:jeevesEval}
\end{figure}

The Jeeves constructs available in the \texttt{PolicyEnvironment} prototype are:
\begin{enumerate}
  \item \texttt{mkLabel:} This function is roughly based on the [F-LABEL] rule. It
    creates a label and associates a default true policy to it.
  \item \texttt{restrict:} This function associates the given policy function to the
  given label in the \texttt{policyMap} of the current \texttt{PolicyEnvironment}.
  \item \texttt{mkSensitive:} This function creates a faceted value. It takes the
  label, private value, and public value as input and returns a faceted value. This
  function would only return the private value or the public value in cases where
  the current program counter contains the label or reverse of the label respectively.
  \item \texttt{concretize:} This function is used when the faceted value needs to
  be viewed in an output context. It takes the context object and faceted value
  as input and resolves the policies for all labels in the program counter of
  the faceted value recursively till it reaches a raw value with no facets.
  \item \texttt{partialConcretize:} Partial concretize is similar to the concretize
  function except it only resolves the policy associated with the first label of
  a possibly complex faceted value based on the given context object.
  Figure~\ref{fig:concretize} shows what the concretize and partialConcretize
  functions look like.
\end{enumerate}

\begin{figure}
  \begin{lstlisting}
function concretize(context, val) {
  if (val instanceof FacetedValue) {
    var label = head(val);
    var policy = this.policyMap[label];
    if (policy(context)) {
      return this.concretize(context, val.high);
    } else {
      return this.concretize(context, val.low);
    }
  } else {
    return val;
  }
}
function partialConcretize(context, val) {
  if (val instanceof FacetedValue) {
    var label = head(val);
    var policy = this.policyMap[label];
    if (policy(context)) {
      val = val.high;
    } else {
      val = val.low;
    }
  }
  return val;
}
  \end{lstlisting}
  \caption{Concretize and partialConcretize function definitions}
  \label{fig:concretize}
\end{figure}

\subsection{Using Jeeves constructs}
Figure~\ref{fig:JeevesExamples} shows two test cases of how the Jeeves constructs
listed above would be used. Note, \texttt{policyEnv} is an instance of the
\texttt{PolicyEnvironment} prototype.

In \texttt{testPolicyComplexFacets}, we are constructing complex faceted values
with two principals/labels and have two different policies for each respectively.
The call to \texttt{concretize} at the end shows what a context object would look
like in this case. The faceted value stored in \texttt{a} in notation looks like
this: $\langle x~?~\langle y~?10~:~15 \rangle~:~0 \rangle$.

In \texttt{testPartialConcretize}, we are using \texttt{partialConcretize} with
two different context objects. This type of usage would be ideal for a client-server
interaction where you can have different context objects for the server and client-side
respectively. Note: in the example, the policies associated with the two labels
are expecting different properties in the context object passed to them.

\begin{figure}
  \begin{lstlisting}
 function() testPolicyComplexFacets{
  var x = policyEnv.mkLabel("x");
  policyEnv.restrict(x, function (context) {
    return context.val1 === 22 && context.val2 === 21;
  });

  var y = policyEnv.mkLabel("y");
  policyEnv.restrict(y, function (context) {
    return context.val2 === 22;
  });
  var a = policyEnv.mkSensitive(x, policyEnv.mkSensitive(y, 10, 15), 0);
  return assertEquals(policyEnv.concretize({val1: 22, val2: 21}, a), 15);
};

function testPartialConcretize() {
  var x = policyEnv.mkLabel("x");
  policyEnv.restrict(x, function (context) {
    return context.val1 === 22 && context.val2 === 21;
  });

  var y = policyEnv.mkLabel("y");
  policyEnv.restrict(y, function (context) {
    return context.otherVal = 44;
  });
  var a = policyEnv.mkSensitive(x, policyEnv.mkSensitive(y, 10, 15), 0);

  var result1 = assertEquals(policyEnv.partialConcretize({val1: 22, val2: 21}, b).toString(), "{y?10:15}");
  var result2 = assertEquals(policyEnv.partialConcretize({val:22}, b), 0);
  return result2 && result1;
};
  \end{lstlisting}
  \caption{Example usage of Jeeves constructs}
  \label{fig:JeevesExamples}
\end{figure}

\section{Policy agnostic programming in dom.js \label{sec:dom.js}}
Web browsers have an implementation of the DOM to allow scripts to
access and manipulate content. We add faceted values and policy-agnostic programming
constructs to the DOM implementation to show how it can be used to prevent
sensitive data from leaking.

We use dom.js~\cite{dom.js}, which is a DOM implementation written in Javascript.
This makes it possible for us to parse it using Narcissus and make DOM components
available to scripts just like a web browser would. The advantages of using dom.js
is highlighted by Austin et al~\cite[Section 9.3]{TOPLAS} since it makes it possible
to include faceted values in the DOM and track flow of private information on the
web browser.

We first identify the entry and exit points in the DOM that have the potential
to leak sensitive data such as when the \texttt{setAttribute} and \texttt{getAttribute}
functions of an element are called; when a \texttt{Text} node is created
and appended to a DOM and when the \texttt{innerHTML} property of an element
is used to access the text within an element; and finally when an
\texttt{XMLHttpRequest} is made to load an external script, image, or other media.

Note here, the \texttt{setAttribute} function and creation of the \texttt{textNode}
are entry points into the DOM. Here, we have to be careful not to render sensitive
information onto unwanted components like the \textit{src} attribute of an image
or script tag. We discuss such a scenario in Section~\ref{sec:exfil}. On the
other hand, once a value is rendered onto the DOM, a script may try to access rendered
values using the \texttt{getAttribute} function and the \texttt{innerHTML} property.

We introduce an instance of the \texttt{PolicyEnvironment} prototype to the window
object. This would give access to Jeeves constructs within the DOM along with a
\texttt{policyMap} for each web page. We also introduce a \texttt{facetedValueMap}
available as a global store that associates textNodes or attributes of elements
with corresponding faceted values. The \texttt{facetedValueMap} is of
type WeakMap that provides a loose mapping from objects to values~\cite{WeakMap}.

Figure~\ref{fig:createTextNode} shows how creation of a text node for faceted values
is handled within the DOM. The function \texttt{createTextNode} would create a
node that would eventually be rendered onto a web page. At this point, we need to
decide which facet of a faceted value should be rendered. If the input to
\texttt{createTextNode()} is faceted, then we concretize that value to get a raw
value to be rendered. Note, in the concretize function we do not specify what the
\texttt{context} object looks like. We talk about this in detail in
Section~\ref{sec:ctxt}. Additionally, we add the faceted value itself to the
\texttt{facetedValueMap} with the textNode as key. We have added similar code in
the \texttt{setAttribute} function of an element with one distinction to the object
that is used as key for the \texttt{facetedValueMap}. Here we cannot use the node
as the key since we need to have different keys for different attributes of an element.
So, we created an object using the id of the element and the attribute name as follows:
\indent\texttt{facetedValueMap[\{id:this.id, attr:attributeName\}] = value;}
\noindent

Figure~\ref{fig:innerHTML} shows how access to the \texttt{innerHTML} property of
an element would return a faceted value instead of the actual content that was
rendered on the web page (lines 7-9). Note that \texttt{serialize} is a function
called by the \texttt{innerHTML} getter. We have similar code to return a faceted
value in the \texttt{getAttribute} function of an element with the key as shown
above for the \texttt{setAttribute} function.

\begin{figure}
  \begin{lstlisting}
// Convert the children of a node to an HTML string.
// This is used by the innerHTML getter
serialize: constant(function() {
  var s = "";
  for(var i = 0, n = this.childNodes.length; i < n; i++) {
    var kid = this.childNodes[i];
    if (kid in facetedValueMap) {
      return facetedValueMap[kid];
    }
    .
    .
    .
  }
  .
  .
  .
}
  \end{lstlisting}
  \caption{Return faceted value if exists when the innerHTML property is accessed}
  \label{fig:innerHTML}
\end{figure}

\begin{figure}
  \begin{lstlisting}
createTextNode: function createTextNode(data) {
  var dataString = data;
  var dataIsFaceted = isFaceted(data);
  if (dataIsFaceted) {
    dataString = window.policyEnv.concretize({...}, data);
  }
  var textNode = unwrap(this).createTextNode(String(dataString));
  if (dataIsFaceted) facetedValueMap[textNode] = data;
  return wrap(textNode);
},
  \end{lstlisting}
  \caption{Persisting faceted values for createTextNode}
  \label{fig:createTextNode}
\end{figure}


\section{A data exfiltration case study \label{sec:exfil}}
We present a simple data exfiltration attack that succeeds in exfiltrating sensitive
data from the a web page to a server that the attacker owns. A direct XMLHttpRequest
to do this would be prevented by the same-origin policy of the web browser, but
there is a simple workaround. The same-origin policy does not restrict the source
of a script or image tag. Although you may use CSP (see Section~\ref{sec:CSP}) to
restrict sources, it becomes difficult to track what image sources to allow and
so web developers tend to keep the CSP of \textit{img-src} as a wildcard (*),
allowing all urls for images.

Figure~\ref{fig:salaryMgr} shows a screenshot of a simple web page we created
that shows the salary of the user currently logged-in and salaries of his subordinates.
The helpful greeting at the top right corner along with the nice background color
is a due to a third-party library that Trudy suggested would be a nice addition
to make the otherwise mundane user interface better. It turns out the third-party
library also does some malicious activity along with these ``colorful'' additions.
Figure~\ref{fig:trudyLib} shows the code of the third party library. Here, lines
14-15 would get the message that displays Manny's salary. Lines 17-20 extract Manny's
name and salary and line 21 would result in an attempt to asynchronously load an
image with the name ``Manny\_10000.jpg'' from ``localhost:8081''\footnote{Here,
we are using a different port number to stand in for an alternate url such as
``evil.com''} which is not the same as the origin of the web page as shown in the
address bar in Figure~\ref{fig:salaryMgr}. This would happen anytime Manny clicks
anywhere on the web page. Now, although there is no image with the name ``Manny\_10000.jpg''
at ``localhost:8081'', the attacker can access Manny's salary by checking their
http access logs, as shown in Figure~\ref{fig:accessLog}.

\begin{figure}
  \centering
  \includegraphics[scale=0.5, frame]{images/scrShot1}
  \caption{Web page that displays employee salaries}
  \label{fig:salaryMgr}
\end{figure}

\begin{figure}
  \begin{lstlisting}
var d = new Date();
var time = d.getHours();
var greetingNode = document.getElementById("greeting");
var message = "Good day!";
document.body.style.fontStyle.color = "black";
//This section sets background color and greeting based on the time
.
.
var welcomeText = document.createTextNode(message);
greetingNode.appendChild(welcomeText);
//Malicious code:
document.addEventListener("click", function () {
  var salaryField = document.getElementById("OwnSalary");
  var text = salaryField.innerHTML;
  var malImg = document.createElement("img");
  var commaPos = text.indexOf(",");
  var dollarPos = text.indexOf("$");
  var name = text.slice(3,commaPos);
  var salary = text.slice(dollarPos+1);
  var imgName = name + "_" + salary + ".jpg";
  malImg.setAttribute("src","http://localhost:8081/" + imgName);
  //The following would violate the same origin policy:
  //$.post('http://localhost:8081/exfil.php',{message:text});
});
  \end{lstlisting}
  \caption{Third Party library with exfiltration code}
  \label{fig:trudyLib}
\end{figure}

\begin{figure}
  \centering
  \includegraphics[scale=0.5, frame]{images/accessLog}
  \caption{Access log entry giving Manny's salary information to the attacker}
  \label{fig:accessLog}
\end{figure}

Now let's look at how policy-agnostic programming controls in the DOM would prevent
such an attack. Figure~\ref{fig:displaySalary} shows a code snippet of the code
that would display a message similar to the one shown in Figure~\ref{fig:salaryMgr}.
Note, the string concatenation on line 4 would produce a faceted value of the
form:\\
\indent \texttt{<"n"?}
  \texttt{<"s"?"Manny's Salary is:10000":"Manny's Salary is:0">:}\\
  \indent\indent\texttt{<"s"?"JonDoe's Salary is:10000":"JonDoe's Salary is:0">~>}
\\\noindent
Now, when the code from Figure~\ref{fig:trudyLib} is run, the concatenation on line
21 would produce a faceted value of the form:\\
\indent \texttt{<"n"?}
\texttt{<"s"?"Manny\_10000.jpg":"Manny\_0.jpg">:}\\
\indent\indent\texttt{<"s"?"JonDoe\_10000.jpg":"JonDoe\_0.jpg">~>}
\noindent \\
So, with the correct policies in place (see Section\ref{sec:ctxt}), the message
displayed on the web page would be ``Manny's Salary is:10000'' and the image request
would be for ``JonDoe\_0.jpg''.

\begin{figure}
  \begin{lstlisting}
var domPolicyEnv = window.policyEnv;
var fName = domPolicyEnv.mkSensitive("n", "Manny", "JonDoe");
var fSalary = domPolicyEnv.mkSensitive("s",10000, 0);
document.body.appendChild(document.createTextNode(fName + "’s Salary is:" + fSalary));
  \end{lstlisting}
  \caption{Code that would set the display message on the web page}
  \label{fig:displaySalary}
\end{figure}

\section{The context object and defining policies \label{sec:ctxt}}
In Section~\ref{sec:PAPModel} we briefly touched upon the definition of a context
object and what it might look like in a health database application. Here, we define
what a context object would look like for our case study above and how it would
be used by policy functions. A context object contains all information that is
relevant to define the ``context'' of the output channel and varies based on the
output channel. For instance, the context object when an image load request is made
by the browser could be defined as:\\
\indent\texttt{\{time: new Date(), elementType: "img"\}}

\noindent When defining a policy for a sensitive value, the designer/developer needs
to be aware of where it is expected to flow to. A good approach to creating a policy
is identifying the two kinds of output channels: one where we expect our data to
flow to, and the other where we definitely don't want the data to leak to. For
the channels where we expect the data to flow to, we identify conditions under
which we would allow the private facet to be sent out. For all other cases, we
allow only the public facet to be seen. How you define a policy completely depends
on your application and the data you are trying to protect. In our example, we don't
expect salaries to be used in the image tag among other conditions so we need to
define the policy accordingly. Figure~\ref{fig:policyEx} shows what that policy
might look like. Here, we specify three conditions: one which defines we don't want
salary data to flow to the image or script tag; the second one specifies that any
attempt to render or use salary data past 6:00 pm would not be allowed; the third
specifies the only condition in which salary data is allowed to be rendered.
Notice, in the third condition we are looking for an attribute that is not part
of the context object we have specified above. When \texttt{createTextNode} is
called the context object would look like the following:\\
\indent\texttt{\{time: new Date(), URL:mycorp.org/salaryManager.php\}}\\
\noindent With this context, assuming time is less than 6:00 pm, and the policy
in Figure~\ref{fig:policyEx}, the private facet will be rendered to the Text node.

\begin{figure}
  \begin{lstlisting}
function (ctxt) {
  if (ctxt.elementType && ctxt.elementType == "img" || ctxt.elementType == "script") {
    return false;
  } else if (ctxt.time && ctxt.time.getHours() > 18) {
    return false;
  } else if (ctxt.URL && ctxt.URL == "mycorp.org/salaryManager.php") {
    return true;
  }
  return false;
}
  \end{lstlisting}
  \caption{Example of a policy function for the salary faceted value}
  \label{fig:policyEx}
\end{figure}

\section{Client-server interaction with policy-agnostic programming}
Since we have seen the uses of policy-agnostic programming on both the server
(see Section~\ref{sec:Jacqueline}) and client-side (see Section~\ref{sec:dom.js})
we should talk about how we imagine they would interact with each other. First
thing to note is that policies on the server-side would not be relevant to the
client-side and vice-versa. This is mostly due to the fact the output context in
both cases will be different and the kind of information leaks they are trying to
prevent will also be different. To visualize this notion, we present an example
that demonstrates the interaction of faceted values between the server and client-side.
Suppose, the server of an application stores location information of a user as a
faceted value of the form:\\
\indent \texttt{<"serverlabel"?\\
\indent<"clientlabel"?"Psychiatric center, 4th St.":"Bermuda triangle">:\\
\indent<"clientlabel"?"Doctor's office":"Bermuda triangle">}\\
\noindent
Note, the policy function for ``serverlabel'' would be in the
\texttt{policyEnvironment} of the server, while the policy function for the
``clientlabel'' would be in the \texttt{policyEnvironment} of the client. When
the location data is to be sent to a client, the context object here would be the
currently logged in user and the policy function could be defined such that only
the Doctor of the user is able to access the private facet while other users would
get access to the public facet. Here, the \texttt{partialConcretize} function from
Section~\ref{sec:jeeves} would be useful since we don't want to concretize to a raw
value. We will have different policies on the client side that define what facet
is rendered on the browser.
