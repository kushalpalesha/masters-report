\chapter{Introduction\label{chap:intro}}

The rapid increase in the number of applications that collect and process private
data has made prevention of data leaks an involving task for security professionals.
It is hard enough protecting against attacks on web servers, we now have sophisticated
web applications that do more than just display static information received from
the server. Javascript is the language of choice to develop these dynamic web pages,
but there is lot of fragmentation in the Javascript engines used by different browsers,
so if one browser may prevent some form of attack, we cannot assume that another
browser may prevent the same form of attack.
%TODO:fix
% Moreover, the flexibility that client-side
% Javascript provides to manipulate the Document Object Model (DOM) makes it challenging
% to secure content on the browser.

In this paper, we propose the introduction of policy-agnostic programming (PAP)
into Javascript to help protect sensitive data on the client-side. PAP is a
programming paradigm introduced by Yang et al.~\cite{Jeeves} that builds on
research efforts in language based information flow security. In their paper,
Yang et al. introduced Jeeves, a language to write policy-agnostic programs. The
PAP paradigm aims to make the implementation of information flow controls in complex
real world systems flexible and intuitive. Yang et al.~\cite{Jacqueline} presented
how PAP can be used to protect data on a database backed server. They introduced
an MVC framework called Jacqueline, which extends the Django framework~\cite{django}
with Jeeves for policy-agnostic evaluation. Similarly we will extend Javascript
to support PAP and demonstrate how it can help prevent sensitive data leaks on the
browser.

In the rest of this chapter we show a survey of the current state of browser security,
proposals to protect against various client-side attacks and where our solution
would fit into the client-side architecture. In Chapter~\ref{chap:PAP} we give a brief background
about information flow security, policy-agnostic programming, Jeeves, and related
concepts. In Chapter~\ref{chap:solution} we review our solution, provide details
of our implementation, and show a sample application. In the final chapter we conclude.

\section{Efforts to secure browser content}
Ever since the first web browser was introduced~\cite{nexus} in 1990, the kind of
content rendered has evolved dramatically. First it was only static html pages that
that were acquired from web servers. The introduction of the Common Gateway
Interface (CGI) in 1993~\cite{cgi}, added the capability of generating dynamic
web pages based on client requests made by the web browser. "Dynamic" here was
pages that were created by web servers based on user requests and hence served
personalized content that may include sensitive data. This meant there was content
worth protecting but even then most of the security measures were focused
on the server-side (where all the data resided) since that was where attackers
also focused their attention. With the introduction of Javascript in 1995~\cite{js10days}
web browsers became really powerful since it enabled web developers to create web
pages with client-side interactivity without the need to make requests to a web server.
% and ``with great power comes great responsibility'' should have been the motto
% back then with regards to security but Spiderman was not out at the time

With further iterations of Javascript and the technologies around it, web pages have
now evolved into web applications giving great control over sensitive user data to the
client-side. The flexibility of Javascript that makes it possible to develop sophisticated
web applications also makes content rendered by web browsers vulnerable to attacks.
Over the years browser vendors have come up with various security measures to
protect browser content.

\subsection{Common security measures found in most modern browsers}
Flanagan~\cite[Section~13.6]{flanagan2011javascript} talks about browser security
in brief and states two competing goals that browser vendors have tried to balance:
``Defining powerful client-side APIs to enable useful web applications.'' and
``Preventing malicious code from reading or altering your data, compromising your
privacy, scamming you, or wasting your time.'' Flanagan~\cite[Section~13.6.1]{flanagan2011javascript}
lists some of the common security restrictions imposed on Javascript and notes
that ``Different browsers have different security policies and may implement
different API restrictions.''.

\subsubsection{Same-origin policy}

\begin{quotation}
  ``The same-origin policy restricts how a document or script loaded from one origin
  can interact with a resource from another origin. It is a critical security mechanism
  for isolating potentially malicious documents.''~\cite{sameOrigin}
\end{quotation}
The article~\cite{sameOrigin} gives details about how the same-origin policy controls
Javascript behavior for different scenarios like cross-origin network access (control
http requests or resource embedding tags), cross-origin script API access (limit
access to Window and Location objects) and cross-origin data storage access. The
same origin policy is very broad in terms of what it controls primarily because it
needs to keep the flexibility that Javascript is known for. Flanagan~\cite[Section~13.6.2]{flanagan2011javascript}
has more details about the same origin policy and lists some techniques of how the
same origin policy can be relaxed in some cases (read Cross-Origin Resource Sharing
(CORS)~\cite{CORS}).
% Later in the report we
% will see a particular exploit that is possible since the same origin policy does
% not restrict loading images from external sources.

\subsubsection{Content Security Policy \label{sec:CSP}}

\begin{quotation}
``Content Security Policy (CSP) is an added layer of security that helps to detect
and mitigate certain types of attacks, including Cross Site Scripting (XSS) and
data injection attacks. These attacks are used for everything from data theft to
site defacement or distribution of malware.''~\cite{csp}
\end{quotation}
In CSP, inline scripts are disabled by default, which would automatically prevent
code injection based attacks. Additionally, CSP also allows a server administrator
to restrict what sources a web page can import executable scripts from. On the face
of it, CSP seems to provide a robust mechanism to prevent XSS and data exfiltration
attacks.

\subsection{Arguments against common browser security mechanisms \label{sec:against}}
The security mechanisms mentioned above are implemented in all modern browsers to
help prevent sensitive data leaks among other security breaches but they are not
foolproof solutions as presented by several researchers:
\begin{itemize}
  \item Chen et al.~\cite{SelfExfil} present cases where the same-origin policy falls short
  in protecting sensitive data from a particular form of the data-exfiltration attack.
  They define a data-exfiltration attack as:

  \begin{quotation}
    ``an attack where the adversary exports user's private data to a server controlled
    by the attacker, possibly using a code injection vulnerability.''
  \end{quotation}
  This form of attack would be prevented by the same origin policy. The authors present
  \textit{self-exfiltration} as a new ``class'' of the data-exfiltration attack. In
  this form of attack, the injected script does not directly send the extracted data
  to an attacker-controlled server; instead it is posted to the another location
  of the victim website itself or to whitelisted origins. The attacker can later
  log-on/access the victim website or whitelisted site respectively to retrieve the
  information.

  \item Acker et al.~\cite{DataExfilCSP} present a strong case arguing the failure of
  CSP to prevent data exfiltration. They show that even the strongest CSP policies
  can be circumvented using DNS and resource prefetching as data exfiltration
  techniques.
\end{itemize}
The researchers above reveal flaws in existing browser security mechanisms and suggest
possible improvements to fix them. As mentioned at the beginning of this chapter,
all browsers are not created equal and so maybe a couple of browser vendors may
implement one of the suggested solutions. But this would mean the vulnerability
would still exist in the rest of the browsers until the said solution becomes a
standard and all the vendors adopt it. Moreover, these are just some examples of
the many loopholes that researchers (and unfortunately attackers) have been finding
in the current browser security architecture.

We propose adding information flow controls to client-side Javascript and use them
to protect content displayed on a web page. We demonstrate its usefulness in
protecting content on the browser by incorporating policy-agnostic programming into
the document object model(DOM)~\cite{DOM}, which is an application programming
interface that defines how programs and scripts can access and update the content,
structure, and style of documents.

%My approach and drawbacks%
