\chapter{Introduction\label{chap:intro}}

The rapid increase in the number of applications that collect and process private
data has made prevention of data leaks an invoving task for security professionals.
It's hard enough protectecting from attacks on web servers, we now have sophisticated
web applications that do more than just display static information received from
the server. Javascript is the language of choice to develop these dynamic web pages
but there is lot of fragmentation in the Javascript engines used by different browsers,
so if one Javascript engine may prevent some form of attack, we cannot assume
that another browser may prevent the same form of attack. Moreover, the flexibilty
that client side Javascript provides to manipulate the Document Object Model (DOM)
makes it challenging to secure content on the browser.

In this paper, we propose the introduction of Policy agnostic programming (PAP)
into Javascript to help protect sensitive data on the client side. PAP is a
programming paradigm introduced by Yang et al in ~\cite{Jeeves} that builds on
research efforts in language based information flow security. In their paper,
Yang et al introduced Jeeves, a language to write policy agnostic programs. The
PAP paradigm aims to make the implementation of information flow controls in complex
real world systems flexible and intuitive. In ~\cite{Jacqueline}, Yang et al
presented how policy agnostic programming can be used to protect data on a database
backed server. They introduced an MVC framework called Jacqueline which extends
the Django framework with Jeeves for policy agnostic evaluation. Similarly we will
extend Javascript to support PAP and demonstrate how it can help prevent sensitive
data leaks on the browser.

In the rest of this chapter we show a survey of the current state of browser security,
proposals to protect against various client-side attacks and where our solution
would fit into the client side architecture. In Chapter 2 we give a brief background
about information flow security, policy agnostic programming, jeeves and related
concepts. In Chapter 3 we give a brief about how our solution would work. In the
subsequent chapters we talk about our results, related work, limitations and future
work. In the final chapter we go into details about of our implementation.

\section{Efforts to secure browser content}
\textbf{Not much content for this section yet. For now this should suffice:}
\textit{The browser is insecure and it is difficult to protect sensitive data.}
